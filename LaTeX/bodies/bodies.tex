\documentclass[landscape]{slides}
\usepackage[landscape]{geometry}
\usepackage[dvips]{graphicx}
\begin{document}
\begin{normalsize}
\begin{slide}
\title{Floating Bodies in \LaTeX{}}
\maketitle
\end{slide}


\begin{slide}
{\Huge What are Floating Bodies?}
\begin{itemize}
\item \emph{Figures} are Floating Bodies.
\item \emph{Pictures} drawn in \LaTeX{} are also Floating Bodies.
\item \emph{Tables} are not always Floating Bodies.
\item Anything can be a Floating Body if you want it to be. 
\end{itemize}
\end{slide}


\begin{slide}
\LaTeX{} treats Floating Bodies differently than text.
\begin{itemize}
\item Floating Bodies are often external files
\item They are not broken accross page breaks.
\item They are inextricably connected to their captions.
\item \TeX{} keeps a running list of Tables and Figures. 
\item Only \TeX{} has direct control of where they are placed.
\end{itemize}
\end{slide}


\begin{slide}
{\huge Creating Floating Bodies}
\vspace{1cm}\\
Tabular Environment\\
 -Table Specifications\\
 -Tables as Floating Bodies
\vspace{1cm}\\
Figure Environment\\
 -Including Figures\\
 -File Formats\\
 -The Graphicx package
\vspace{1cm}\\
Picture Environment\\
 -Drawing pictures within \LaTeX{}\\
\end{slide}

\begin{slide}
{\Huge The Tabular Environment}
\vspace{1cm}\\
Tables are made in an environment all their own.
\vspace{1cm}\\
\verb+\begin[+\emph{pos}\verb+]{tabular}{table specifications}+\\
\verb+\end{tabular}+\\
\vspace{1cm}\\
To float a table:\\
\vspace{1cm}\\
\verb+\begin{table}+\\
\verb+\begin[+\emph{pos}\verb+]{tabular}{table specifications}+\\
\verb+\end{tabular}+\\
\verb+\end{table}+\\
\end{slide}


\begin{slide}
{\Huge The Figure Environment}
\vspace{1cm}\\
In order to use \verb+\includegraphics{}+ you must first call the graphicx package:\\
\verb+\usepackage[dvips]{graphicx}+
\vspace{1cm}\\
The figure environment is a floating body environment.
\vspace{1cm}\\
\verb+\begin[+\emph{pos}\verb+]{figure}{figure specifications}+\\
\verb+\includegraphics{+\emph{filename}\verb+}+\\
\verb+\end{figure}+\\

\end{slide}


\begin{slide}
{\Huge The Picture Environment}
\vspace{1cm}\\
\verb+\begin{picture}(x,y)(+$x_0,y_0$\verb+)+\\
.\\
.\\
.\\
\verb+\put(x,y){object}+\\
.\\
.\\
.\\
\verb+\end{picture}+\\

\end{slide}


\begin{slide}
\begin{center}
{\Huge The Placement of Floating Bodies}
\end {center}
\end{slide}


\begin{slide}
\begin{itemize}
\item \LaTeX{} is egomaniacal
\begin{itemize}
\item \TeX{} feels it knows what looks best.
\item \TeX{} will ignore your commands if they suggest something else.
\end{itemize}
\end{itemize}
\end{slide}


\begin{slide}
\begin{center}
Recall the position specifier \verb+[+\emph{pos}\verb+]+
\vspace{1cm}\\
\verb+\begin[+\emph{pos}\verb+]{figure}{figure specifications}...+
\vspace{1cm}\\
\emph{pos}$ = $ h, hb, ht, t, p, b, !, htbp!...
\vspace{1cm}\\
h = here\\
t = top\\
b = bottom\\
p = independent page\\
! = even if it looks bad\\
\end{center}
\end{slide}


\begin{slide}
{\Huge Boxes}
\vspace{1cm}\\
Boxes for Paragraphs:\\
\verb+\parbox[pos]{width}{+\emph{...text..}\verb+}+\\
\vspace{1cm}\\
Boxes for anything:\\
\verb+\begin{minipage}[pos]{width}+\\
\emph{...text, figures, tables...}\\
\verb+\end{minipage}+\\
\end{slide}
\begin{slide}
{\Huge Boxes Cont'd}
\vspace{1cm}\\
The position specifier for boxes gives slightly more control than before, since \TeX{} does not treat them quite like floating bodies.\\
Boxes are glued to the text with `special glue, which is elastic'.\\
\emph{pos} can be t, c, or b, and refers to the relation of the box to the surrounding text.\\
Also, the width specifier allows graphs and images to be horizontally adjacent to each other.\\
\end{slide}

\begin{slide}
\begin{center}
{\Huge \verb+\clearpage+}\\
\end{center}
\end{slide}


\begin{slide}
\begin{center}
{\Huge \verb+\FloatBarrier+}\\
\end{center}
\end{slide}

\begin{slide}
\begin{center}
fin.
\end{center}
\end{slide}

\end{normalsize}
\end{document}

